\documentclass{article}
\title{Project Euler Problem 17\\Number Letter Counts}
\author{Patrick Harrington}
\usepackage{amsmath}
\usepackage{subcaption}

\begin{document}
\maketitle
\section{Problem Statement}
If the numbers 1 to 5 are written out in words: one, two, three, four,
five, then there are 3 + 3 + 5 + 4 + 4 = 19 letters used in total.

If all the numbers from 1 to 1000 (one thousand) inclusive were written out
in words, how many letters would be used?


NOTE: Do not count spaces or hyphens. For example, 342 (three hundred and
forty-two) contains 23 letters and 115 (one hundred and fifteen) contains 20
letters. The use of ``and'' when writing out numbers is in compliance with
British usage.

\section{Method}
Since we are converting each number to a string in the range {1,1000}, we
might as well save ourselves a great deal of work and find common quantities.
\subsection{Redifining the numbers}

The following table shows all of the most common numbers (which also make up
parts of larger numbers) as the integer value of the letters present.

\begin{tabular}{ccc}
  \begin{minipage}{.3\linewidth}
  \begin{tabular}[center]{c|c}
    Number  & Letters \\
    \hline
    One & 3\\
    Two & 3\\
    Three & 5\\
    Four & 4\\
    Five & 4\\
    Six & 3\\
    Seven & 5\\
    Eight & 5\\
    Nine & 4\\
  \end{tabular}
  \end{minipage} &

  \begin{minipage}{.3\linewidth}
  \begin{tabular}[center]{c|c}
    Number  & Letters \\
    \hline
    Ten & 3\\
    Eleven & 6\\
    Twelve & 6\\
    Thirteen & 8\\
    Fourteen & 8\\
    Fifteen & 7\\
    Sixteen & 7\\
    Seventeen & 9\\
    Eighteen & 8\\
    Nineteen & 8\\
  \end{tabular}
  \end{minipage}  &
  
  \begin{minipage}{.3\linewidth}
  \begin{tabular}[center]{c|c}
    Number  & Letters \\
    \hline
    Twenty & 6\\
    Thirty & 6\\
    Forty & 5\\
    Fifty & 5\\
    Sixty & 5\\
    Seventy & 7\\
    Eighty & 6\\
    Ninety & 6\\
  \end{tabular}
  \end{minipage}  
\end{tabular}



\subsection{Summation}
The next logical step was to find the values for repeating ranges such as 1-9,
1-99 and so on. 

\subsubsection{\{ 1-9 \}}
The first range to have its values summed was 1 through 9:
\begin{eqnarray*}
  a &=& \sum_{i=1}^{9} Letters[i]\\
  a &=& \boxed{36}
  \label{1thru9}
\end{eqnarray*}

\subsubsection{ \{ 10-19 \} }
Next, the teens
\begin{eqnarray*}
  b &=& \sum_{i=10}^{19} Letters[i]\\
  b &=& \boxed{70}
  \label{10thru19}
\end{eqnarray*}

\subsubsection{\{ 20-90 \}}
Next, a corner case was considered; every ten integers were summed as \emph{c}.
\begin{eqnarray*}
  c &=& \sum_{i=20}^{90} Letters[i\times10]\\
  c &=& \boxed{46}
  \label{20thru90}
\end{eqnarray*}

So the number of letters in the range \{1,99\} is:
\begin{eqnarray*}
  9a+b+10c &=& d\\ 
  d &=&\boxed{854} 
  \label{1thru99sum}
\end{eqnarray*}

With this in mind, and given that 
\begin{eqnarray*}
  \text{One Thousand} &=& \boxed{11}
\end{eqnarray*}

the final equation can be set up:

\begin{eqnarray*}
  \overbrace{10d}^{\text{1$\to$99 }} +
  \underbrace{(900 \times 7)}_{\text{hundred}} +
  \overbrace{(891\times3)}^{\text{and}}+
  100a+11\\
\end{eqnarray*}                   
\begin{eqnarray*}
  \text{\bf{Final Answer}}: \boxed{21124}
\end{eqnarray*}                   
\end{document}
